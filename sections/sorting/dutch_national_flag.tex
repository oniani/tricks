%%%%%%%%%%%%%%%%%%%%%%%%%%%%%%%%%%%%%%%%%%%%%%%%%%%%%%%%%%%%%%%%%%%%%%%%%%%%%%%
% Dutch National Flag Problem
%%%%%%%%%%%%%%%%%%%%%%%%%%%%%%%%%%%%%%%%%%%%%%%%%%%%%%%%%%%%%%%%%%%%%%%%%%%%%%%

\subsection{Dutch National Flag Problem}

\begin{figure}[H]
    Optimal Solution:\\\\
    \begin{tabular}{rl}
        Time Complexity:& \(O(m)\) where \(m\) is the length of the array.\\
        Space Complexity:& \(O(1)\).
    \end{tabular}
\end{figure}

Given an array nums with n objects colored red, white, or blue, sort them
in-place so that objects of the same color are adjacent, with the colors in the
order red, white, and blue. Integers \(0\), \(1\), and \(2\) are used to
represent the color red, white, and blue, respectively.
\\\\
The first approach performs two passes, but the space complexity is constant.

\begin{figure}[H]
    \centering
    \begin{minted}{python}
        from collections import Counter


        def sort_colors(arr: list[int]) -> None:
            """Sort the list in-place."""
        
            # Counts
            counts: Counter[int] = Counter(arr)

            # Index
            idx: int = 0

            # Store all the 0s in the beginning 
            while (counts[0] > 0): 
                arr[idx] = 0
                counts[0] -= 1
                idx += 1

            # Then all the 1s 
            while (counts[1] > 0): 
                arr[idx] = 1
                counts[1] -= 1
                idx += 1

            # Finally, all the 2s 
            while (counts[2] > 0): 
                arr[idx] = 2
                counts[2] -= 1
                idx += 1
    \end{minted}
\end{figure}

\clearpage

The second approach is more efficient in terms of runtime complexity as it only
performs a single pass. The space complexity is still constant.

\begin{figure}[H]
    \centering
    \begin{minted}{python}
        def sort_colors(arr: list[int]) -> None:
            """Sort the list in-place."""

            ptr1: int = 0
            ptr2: int = len(arr) - 1
            cur: int = 0
            while cur <= ptr2:
                # If `cur` at 0, swap and advance both `cur` and `ptr1`
                if arr[cur] == 0:
                    arr[cur], arr[ptr1] = arr[ptr1], arr[cur]
                    cur += 1
                    ptr1 += 1
                # If `cur` at 2, swap and move only `ptr2` as `ptr2` can be 1
                elif arr[cur] == 2:
                    arr[cur], arr[ptr2] = arr[ptr2], arr[cur]
                    ptr2 -= 1
                # If `cur` at 0, just increment `cur`
                else:
                    cur += 1
    \end{minted}
\end{figure}
