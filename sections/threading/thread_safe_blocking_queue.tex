%%%%%%%%%%%%%%%%%%%%%%%%%%%%%%%%%%%%%%%%%%%%%%%%%%%%%%%%%%%%%%%%%%%%%%%%%%%%%%%
% Thread-safe Blocking Queue
%%%%%%%%%%%%%%%%%%%%%%%%%%%%%%%%%%%%%%%%%%%%%%%%%%%%%%%%%%%%%%%%%%%%%%%%%%%%%%%

\subsection{Thread-safe Blocking Queue}

\begin{figure}[H]
    Optimal Solution:\\\\
    \begin{tabular}{rl}
        Time Complexity:& \(O(1)\) for \mintinline{python}{put},
        \mintinline{python}{get}, and \mintinline{python}{size}.\\
        Space Complexity:& \(O(m)\) where \(m\) is the number of items in the
        queue.
    \end{tabular}
\end{figure}

\begin{figure}[H]
    \centering
    \begin{minted}{python}
        from collections import deque
        from threading import Condition


        class ThreadSafeBlockingQueue():
            """An implementation of the thread-safe blocking queue."""

            def __init__(self, capacity: int) -> None:
                self._capacity: int = capacity
                self._q: deque[int] = deque()
                self._cv: Condition = Condition()

            def put(self, val: int) -> None:
                """Put a value into the queue."""

                with self._cv:
                    # Block
                    while len(self._q) == self._capacity:
                        self._cv.wait()

                    self._q.append(val)

                    # After putting an item, we call `notifyAll` method. Since
                    # we just added an item to the queue, it is possible that a
                    # consumer thread is blocked in the `get` method of the
                    # class, waiting for an item to become available. Hence, we
                    # send a signal to wake up any waiting threads.
                    self._cv.notifyAll()

            def get(self) -> int:
                """Get the value from the queue."""

                with self._cv:
                    # Block
                    while len(self._q) == 0:
                        self._cv.wait()

                    val = self._q.popleft()

                    # After consuming an item, `notifyAll` is called since if
                    # the queue is full, then there might be producer threads
                    # blocked in the `get()` method
                    self._cv.notifyAll()

                    return val

            def size(self) -> int:
                """Return the size of the queue."""

                return len(self._q)
    \end{minted}
\end{figure}
