%%%%%%%%%%%%%%%%%%%%%%%%%%%%%%%%%%%%%%%%%%%%%%%%%%%%%%%%%%%%%%%%%%%%%%%%%%%%%%%
% Backtracking - Generate Permutations
%%%%%%%%%%%%%%%%%%%%%%%%%%%%%%%%%%%%%%%%%%%%%%%%%%%%%%%%%%%%%%%%%%%%%%%%%%%%%%%

\section{Backtracking - Generate Permutations}

\begin{figure}[H]
    Optimal Solution:\\\\
    \begin{tabular}{rl}
        Time Complexity:& \(O(m!)\) where \(m\) is the length of the array.\\
        Space Complexity:& \(O(m!)\) where \(m\) is the length of the array.
    \end{tabular}
\end{figure}

\begin{figure}[H]
    \centering
    \begin{minted}{python}
        def permute(nums: list[int]) -> list[list[int]]:
            """Use backtracking to generate all permutations."""

            # Use Python's implicit "booleannes"
            # `[]` => `False` (generally works for empty sequences)
            if not nums:
                return []

            # The array to be filled
            res: list[list[int]] = []

            # Backtrack and fill the results list
            # NOTE: lists in Python work like pass-by-reference in C++
            backtrack(nums, 0, [], res)

            # Return the result
            return res


        def backtrack(
            nums: list[int], start: int, cur: list[int], res: list[list[int]]
        ) -> None:
            """Perform backtracking on the array of numbers."""

            # Base case: we got one full permutation
            if len(cur) == len(nums):
                # Need to copy since
                res.append(cur[:])
                return

            # Recursive case: generate, recurse, and backtrack
            for i in range(start, len(nums)):
                # Append to the current permutation
                cur.append(nums[i])

                # Place the `i`th element at the start position
                # NOTE: Needed to include previous elements (before `start`)
                nums[i], nums[start] = nums[start], nums[i]

                # Recurse
                backtrack(nums, start + 1, cur, res)

                # Backtrack
                nums[start], nums[i] = nums[i], nums[start]
                cur.pop()
    \end{minted}
\end{figure}
