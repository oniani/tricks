%%%%%%%%%%%%%%%%%%%%%%%%%%%%%%%%%%%%%%%%%%%%%%%%%%%%%%%%%%%%%%%%%%%%%%%%%%%%%%%
% Draw Circle (Trigonometry)
%%%%%%%%%%%%%%%%%%%%%%%%%%%%%%%%%%%%%%%%%%%%%%%%%%%%%%%%%%%%%%%%%%%%%%%%%%%%%%%

\subsection{Draw Circle (Trigonometry)}

\begin{figure}[H]
    Optimal Solution:\\\\
    \begin{tabular}{rl}
        Time Complexity:& \(O(m)\) where \(m\) is the number of iterations.\\
        Space Complexity:& \(O(1)\).
    \end{tabular}
\end{figure}

Given a function that can draw a dot on the screen, write a function that draws
the circle with the given radius.

\begin{figure}[H]
    \centering
    \begin{minted}{python}
        import numpy as np
        import matplotlib.pyplot as plt

        def draw_circle(radius: float, iterations: int) -> None:
            """Draw a circle with a given radius."""

            for angle in range(iterations):
                # Degree -> Radian Formula: 180 * degree / pi
                # But apparently `np.deg2rad` was designed just for this
                a: float = np.deg2rad(angle)
                x: float = radius * np.cos(a)
                y: float = radius * np.sin(a)
                plt.plot(x, y, marker=".")

            plt.title("Trigonometry-based Circle Drawing Algorithm")
            plt.show()
    \end{minted}
\end{figure}
