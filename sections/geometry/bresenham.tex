%%%%%%%%%%%%%%%%%%%%%%%%%%%%%%%%%%%%%%%%%%%%%%%%%%%%%%%%%%%%%%%%%%%%%%%%%%%%%%%
% Bresenham's Circle Drawing Algorithm
%%%%%%%%%%%%%%%%%%%%%%%%%%%%%%%%%%%%%%%%%%%%%%%%%%%%%%%%%%%%%%%%%%%%%%%%%%%%%%%

\subsection{Bresenham's Circle Drawing Algorithm}

\begin{figure}[H]
    Optimal Solution:\\\\
    \begin{tabular}{rl}
        Time Complexity:& \(O(m)\) where \(m\) is the radius of the circle.\\
        Space Complexity:& \(O(1)\).
    \end{tabular}
\end{figure}

Given a function that can draw a dot on the screen, write a function that draws
the circle with the given radius.

\begin{figure}[H]
    \centering
    \begin{minted}{python}
        import matplotlib.pyplot as plt


        def bresenham(radius: float) -> None:
            """Draw a circle using Bresenham's Circle Drawing Algorithm."""

            # Initial x coordinate
            x: int = 0

            # Initial y coordinate
            y: int = radius

            # Decision parameter
            d: int = 3 - 2 * radius

            # While y coordinate is not less than the x coordinate
            while x <= y:
                plt.scatter(
                    [x, x, -x, -x, y, -y, y, -y],
                    [y, -y, y, -y, x, x, -x, -x],
                    marker=".",
                )

                if d <= 0:
                    d += 4 * x + 6
                else:
                    y -= 1
                    d += 4 * (x - y) + 10

                x += 1

            plt.title("Bresenham's Circle Drawing Algorithm")
            plt.show()
    \end{minted}
\end{figure}
