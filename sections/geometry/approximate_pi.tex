%%%%%%%%%%%%%%%%%%%%%%%%%%%%%%%%%%%%%%%%%%%%%%%%%%%%%%%%%%%%%%%%%%%%%%%%%%%%%%%
% Pi Approximation
%%%%%%%%%%%%%%%%%%%%%%%%%%%%%%%%%%%%%%%%%%%%%%%%%%%%%%%%%%%%%%%%%%%%%%%%%%%%%%%

\subsection{\(\pi\) Approximation}

\begin{figure}[H]
    Optimal Solution:\\\\
    \begin{tabular}{rl}
        Time Complexity:& \(O(m)\) where \(m\) is the number of iterations.\\
        Space Complexity:& \(O(1)\).
    \end{tabular}
\end{figure}

Given a function that produces random numbers from the uniform distribution of
[0, 1], estimate \(pi\).

\begin{figure}[H]
    \centering
    \begin{minted}{python}
        import numpy as np

        def pi(iterations: int) -> float:
            """A function for approximating the Pi."""

            circle_area: int = 0
            square_area: int = 0
            for _ in range(iterations):
                # Notice the following:
                #     (1) 0 <= `np.random.uniform()` < 1
                #     (2) 0 <= `2 * np.random.uniform()` < 2
                #     (3) -1 <= `2 * np.random.uniform()`  - 1 < 1
                x: float = 2 * np.random.uniform() - 1
                y: float = 2 * np.random.uniform() - 1
                distance: float = np.sqrt(x ** 2 + y ** 2)
                if distance <= 1:
                    circle_area += 1
                square_area += 1

            return circle_area / square_area * 4
    \end{minted}
\end{figure}
