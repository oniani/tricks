%%%%%%%%%%%%%%%%%%%%%%%%%%%%%%%%%%%%%%%%%%%%%%%%%%%%%%%%%%%%%%%%%%%%%%%%%%%%%%%
% Boyer–Moore Majority Vote Algorithm - Majority Element
%%%%%%%%%%%%%%%%%%%%%%%%%%%%%%%%%%%%%%%%%%%%%%%%%%%%%%%%%%%%%%%%%%%%%%%%%%%%%%%

\section{Boyer–Moore Majority Vote Algorithm - Majority Element}

\begin{figure}[H]
    Optimal Solution:\\\\
    \begin{tabular}{rl}
        Time Complexity:& \(O(m)\) where \(m\) is the length of the array.\\
        Space Complexity:& \(O(1)\).
    \end{tabular}
\end{figure}

\begin{figure}[H]
    \centering
    \begin{minted}{python}
    def majority_element(nums: list[int]) -> int:
        """Find the majority element in the array (element that occurs more
           than `m` times where `m` is the length of the array).
        """

        candidate: int = 0
        count: int = 0
        for num in nums:
            # Candidate gets change iff `count == 0`
            candidate = candidate if count else num
            # Count gets incremented if the candidate and the current number
            # are the same and decrement otherwise
            count += 1 if candidate == num else -1

        return candidate
    \end{minted}
\end{figure}
