%%%%%%%%%%%%%%%%%%%%%%%%%%%%%%%%%%%%%%%%%%%%%%%%%%%%%%%%%%%%%%%%%%%%%%%%%%%%%%%
% Levenshtein Distance - Edit Distance
%%%%%%%%%%%%%%%%%%%%%%%%%%%%%%%%%%%%%%%%%%%%%%%%%%%%%%%%%%%%%%%%%%%%%%%%%%%%%%%

\section{Levenshtein Distance -- Edit Distance}

\begin{figure}[H]
    Optimal Solution:\\\\
    \begin{tabular}{rl}
        Time Complexity:& \(O(m \times n)\) where \(m\) is the number of rows
        and \(n\) is the number of columns.\\
        Space Complexity:& \(O(m \times n)\) where \(m\) is the number of rows
        and \(n\) is the number of columns.
    \end{tabular}
\end{figure}

Given two strings \mintinline{python}{word1} and \mintinline{python}{word2}, we
need to find the optimal/minimum number of operations required to convert
\mintinline{python}{word1} to \mintinline{python}{word2}. We are permitted the
following four operations:

\begin{enumerate}
    \item Insert a character
    \item Delete a character
    \item Replace a character
    \item Keep a character
\end{enumerate}

We take the bottom-up dynamic programming approach:

\begin{figure}[H]
    \centering
    \begin{minted}{python}
        def levenshtein_distance(word1: str, word2: str) -> int:
            """Computes the Levenshtein's distance between two words."""

            # Get the lengths
            m: int = len(word1)
            n: int = len(word2)

            # Prepopulate the DP matrix
            dp: list[list[int]] = [[0] * (n + 1) for _ in range(m + 1)]

            # Initialize the DP matrix with base cases
            for i in range(1, m + 1):
                dp[i][0] += i
            for j in range(1, n + 1):
                dp[0][j] += j

            # Fill the DP Matrix
            for i in range(1, m + 1):
                for j in range(1, n + 1):
                    left: int = dp[i][j - 1]
                    top: int = dp[i - 1][j]
                    top_left: int = dp[i - 1][j - 1]

                    # If characters are equal, just copy the diagonal value
                    if word1[i - 1] == word2[j - 1]:
                        dp[i][j] = top_left
                    # Otherwise, take the minimum of three adjacent cells and
                    # increment the result by one
                    else:
                        dp[i][j] = min(left, top, top_left) + 1

            # Return the optimal solution
            return dp[m][n]
    \end{minted}
\end{figure}
