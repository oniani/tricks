%%%%%%%%%%%%%%%%%%%%%%%%%%%%%%%%%%%%%%%%%%%%%%%%%%%%%%%%%%%%%%%%%%%%%%%%%%%%%%%
% Trie (A Prefix Tree)
%%%%%%%%%%%%%%%%%%%%%%%%%%%%%%%%%%%%%%%%%%%%%%%%%%%%%%%%%%%%%%%%%%%%%%%%%%%%%%%

\subsection{Trie (A Prefix Tree)}

There are many ways to constuct a trie. We will use a hash map based approach.
We first create a \mintinline{python}{Node} class.

\begin{figure}[H]
    \centering
    \begin{minted}{python}
        class TrieNode:
            def __init__(self) -> None:
                """Initialize the `TrieNode` class."""

                self._children: dict[str, TrieNode] = {}
                self._isendofword: bool = False
    \end{minted}
\end{figure}

We can now design the trie data structure. The initializer will only contain a
root of the trie.

\begin{figure}[H]
    \centering
    \begin{minted}{python}
        class Trie:
            def __init__(self) -> None:
                """Initialize the `Trie` class."""

                self._root = TrieNode()
    \end{minted}
\end{figure}

Now, the first method to implement is the insertion operation. If we cannot
insert a word into a trie, we are not going to be able to do much with it. The
algorithm is very straightforward. One iterates over all characters in the word
to be inserted and checks if the characters is in the children of the current
node. If it is, follow the path of that node. If not, create a new node at that
position and once again, follow that road.

\begin{figure}[H]
    \centering
    \begin{minted}{python}
        def insert(self, word: str) -> None:
            """Initialize the `Trie` class."""

            ptr: Node = self._root
            for char in word:
                if char not in ptr._children:
                    ptr._children[char] = TrieNode()
                ptr = ptr._children[char]

            ptr._isendofword = True
    \end{minted}
\end{figure}

One of the primary features of the trie is to be able to search for a word.
The implementation of this algorithm is shown below.

\begin{figure}[H]
    \centering
    \begin{minted}{python}
        def search(self, word: str) -> bool:
            """Check if a trie contains a given word."""

            ptr: Node = self._root
            for char in word:
                if char not in ptr._children:
                    return False
                ptr = ptr._children[char]

            return ptr._isendofword
    \end{minted}
\end{figure}

Similarly, searching a prefix is also very important. It looks almost exactly
the same as the code for searching a word.

\begin{figure}[H]
    \centering
    \begin{minted}{python}
        def prefix_search(self, word: str) -> bool:
            """Check if a trie contains a given prefix."""

            ptr: Node = self._root
            for char in word:
                if char not in ptr._children:
                    return False
                ptr = ptr._children[char]

            return True
    \end{minted}
\end{figure}
