%%%%%%%%%%%%%%%%%%%%%%%%%%%%%%%%%%%%%%%%%%%%%%%%%%%%%%%%%%%%%%%%%%%%%%%%%%%%%%%
% Flood Fill - Number of Islands
%%%%%%%%%%%%%%%%%%%%%%%%%%%%%%%%%%%%%%%%%%%%%%%%%%%%%%%%%%%%%%%%%%%%%%%%%%%%%%%

\subsection{Flood Fill -- Number of Islands}

\begin{figure}[H]
    Optimal Solution:\\\\
    \begin{tabular}{rl}
        Time Complexity:& \(O(m \times n)\) where \(m\) is the number of rows
        and \(n\) is the number of columns.\\
        Space Complexity:& \(O(m \times n)\) where \(m\) is the number of rows
        and \(n\) is the number of columns.
    \end{tabular}
\end{figure}

\begin{figure}[H]
    \centering
    \begin{minted}{python}
        def islands(grid: List[List[str]]) -> int:
            """Counts the number of islands of 1s.

            NOTE: Uses the fact that lists are mutable (will not work in purely
                  functional languages such as Haskell).
            """

            m: int = len(grid)
            n: int = len(grid[0])

            if m * n == 0:
                return 0

            islands: int = 0
            for i in range(m):
                for j in range(n):
                    if grid[i][j] == "1":
                        self.fill(grid, i, j)
                        islands += 1

            return islands


        def flood_fill(grid: list[list[int]], i: int, j: int) -> bool:
            """Implements the Flood Fill algorithm."""

            # Row index out of bounds
            if not (0 <= i <= len(grid) - 1):
                return False

            # Column index out of bounds
            if not (0 <= j <= len(grid[0]) - 1):
                return False

            # If the current symbol is not 1, no need to continue
            if grid[i][j] != "1":
                return False

            # Mark with a sentinel symbol
            grid[i][j] = "$"

            # Recurse in four directions
            flood_fill(grid, i - 1, j)
            flood_fill(grid, i + 1, j)
            flood_fill(grid, i, j - 1)
            flood_fill(grid, i, j + 1)
    \end{minted}
\end{figure}
