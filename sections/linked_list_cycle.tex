%%%%%%%%%%%%%%%%%%%%%%%%%%%%%%%%%%%%%%%%%%%%%%%%%%%%%%%%%%%%%%%%%%%%%%%%%%%%%%%
% Floyd's Cycle Finding Algorithm - Linked List Cycle Detection
%%%%%%%%%%%%%%%%%%%%%%%%%%%%%%%%%%%%%%%%%%%%%%%%%%%%%%%%%%%%%%%%%%%%%%%%%%%%%%%

\section{Floyd's Cycle Finding Algorithm -- Linked List Cycle Detection}

\begin{figure}[H]
    Optimal Solution:\\\\
    \begin{tabular}{rl}
        Time Complexity:& \(O(m)\) where \(m\) is the number of nodes.\\
        Space Complexity:& \(O(m)\) where \(m\) is the number of nodes.
    \end{tabular}
\end{figure}

\begin{figure}[H]
    \centering
    \begin{minted}{python}
        from typing import Union


        class Node:
            def __init__(self, val: int = 0, nxt: Node = Union[None, Node]):
                self._val = val
                self._next = nxt


        def floyd_cycle(head: Node) -> bool:
            """An implementation of Floyd's Cycle Finding Algorithm."""

            if not head:
                return False

            # Initialize slow and fast pointers
            slow: Node = head
            fast: Node = head.next
            while slow != fast:
                # No need to check if `fast._next._next`
                # We check only `fast._next` so that `fast = fast._next._next`
                # does not result in an error
                if not fast or not fast._next:
                    return False

                slow = slow._next
                fast = fast._next._next

            return True
    \end{minted}
\end{figure}
